%Documentation for Final project
%Language:‌ Persian
\documentclass[oneside]{report}
   \usepackage[
   left = 2.5cm ,
   right = 3.5cm,
   bottom = 3cm,
   ]{geometry}
\usepackage{titlesec}

\usepackage[extrafootnotefeatures]{xepersian}

\settextfont{B Nazanin}
\linespread{1.3}
\setlength{\skip\footins}{2cm}
\setlength{\footnotesep}{0.3cm}

\setcounter{secnumdepth}{4}%bound for subsections
\setcounter{tocdepth}{4}

\titleformat{\chapter}[display]
	{\normalfont\Huge\bfseries\centering}
	{\chaptertitlename \thechapter}{20pt}{\Huge\bfseries}
\setlatintextfont{Georgia}

\begin{document}
 
 \section*{چکیده}
\Large \noindent
با توجه به مشکلات امنیتی موجود در زمینه‌ی پرداخت الکترونیکی ، استفاده از راه‌حل‌ کیف پول الکترونیکی به منظور رفع این مشکلات حائز اهمیت است. به سبب اینکه پرداخت‌های امروزی ، اطلاعات کارت بانکی را در معرض خطر قرار می‌دهند، پنهان بودن اطلاعات پرداخت و امن‌تر کردن تراکنش‌ها به علاوه‌ی سهولت پرداخت ،از اهدف اصلی کیف پول ‌الکترونیکی می‌باشد.

\noindent
در فصل‌های ابتدایی تکنولوژی‌
	{\large \lr	{Tokenisation}}
مورد مطالعه قرار خواهد گرفت که نقش اصلی در ایمن‌سازی کیف‌پول‌های الکترونیکی را دارد و در باقی فصل‌ها به بررسی بخش‌های مختلف پروژه‌ی پیاده‌سازی شده پرداخته می‌شود.

\noindent
پروژه‌ی انجام شده پیاده‌سازی مدلی از پرداخت است که با کمک تکنولوژی
{\large \lr {iBeacon}}
اپل به منظور پیدا کردن موقعیت کاربر مورد استفاده قرار گرفته است.


\vspace*{1cm}
\noindent
\textbf{کلید واژگان: } 
{\large \lr	{Tokenisation}} ، {\large \lr {iBeacon}} 

\pagenumbering{gobble}
{\let\cleardoublepage\clearpage 
	\tableofcontents
}
	%%%%%%%%%%%%%%%%%%%%%%%%%%%%%%%%%%%%%%%%%%%%%%%%%%%%%%
	\chapter{مقدمه}\label{introduction}
		\pagenumbering{arabic}
		امروزه اکثر خرید‌ها و به موجب آن پرداخت‌ها از طریق کارت‌های بانکی صورت می‌گیرد، به همین خاطر توجه سارقان و کلاهبرداران این حوزه را به خود جلب کرده است. سرویس‌های کیف‌پول الکترونیکی این اجازه را به کاربران می‌دهند که کارت‌های بانکی خود را بصورت دیجیتالی در دستگاه‌های دیجیتالی مانند تلفن‌های هوشمند ذخیره کنند. بدین طریق کاربران در زمان پرداخت به جای استفاده از کارت‌های بانکی حقیقی از کارت دیجیتالی ذخیره شده بر روی دستگاه دیجیتالیشان استفاده می‌کنند.تدابیر امنیتی مناسبی برای ایمن کردن اینگونه پرداخت‌ها دیده شده که  مهم‌ترین آن‌ها تکنولوژی  {\large \lr	{Tokenisation}}  می‌باشد. در فصل \ref{tokenisation} به توضیح این تکنولوژی پرداخته خواهد شد.
		
		برای خرید یک کالا و یا یک سرویس مدل‌های مختلفی وجود دارد که کاربران می‌توانند با استفاده از آن‌ها پرداخت را انجام دهند. برای مثال کاربران می‌توانند خریدهایشان را از راه ترمینال‌های مجهز به تکنولوژی 
			{\large \lr	{NFC}}
		\LTRfootnote{Near Field Communication}
			  ، و یا از راه نرم‌افزار‌های پرداخت به راه‌های مختلف انجام دهند. در ادامه‌ی این فصل به چند نمونه از مدل‌های پرداخت که در حال حاضر موجود هستند خواهیم پرداخت.		
			  
			  لازم به ذکر است ، کیف پول الکترونیکی که به منظور نگهداری بیت کوین‌ها
			  \LTRfootnote{Bitcoin}
			  و یا پول الکترونیکی می‌باشد مورد بحث نخواهد بود.
			  
				\section{معرفی سرویس‌های برجسته‌ی کیف پول الکترونیکی }
		
		از جمله‌ی سرویس‌های کیف‌پول ،می‌توان 
				{\large \lr{Apple Pay}} 
		را نام برد که اپل در ماه سپتامبر ۲۰۱۴ با معرفی 
				{\large \lr{iPhone 6}} 
		آن را عرضه کرد.	شرکت گوگل نیز سرویس 
						{\large \lr{Android Pay}} 
		را در ماه فوریه‌ی ۲۰۱۵ معرفی کرد. سرویس‌های کیف‌پولی که این شرکت‌ها ارائه کرده اند، بر پایه‌ی تکنولوژی 
			{\large \lr	{Tokenisation}}
			می‌باشند، و از این تکنولوژی برای مدیریت کارت‌های بانکی استفاده می‌کنند. لازم به ذکر است که سرویس 
						{\large \lr{Apple Pay}} 
						به عنوان امن‌ترین روش پرداخت موجود در جهان شناخته شده است. 
						\cite{mostsecureBellID}
			در فصل \ref{comparison} به مقایسه‌ی دو سرویس ارائه شده توسط این دو شرکت  و علت برتری سرویس
										{\large \lr{Apple Pay}} 
			پرداخته خواهد شد.
		
		\subsection{ سرویس {\normalsize\lr{Apple Pay}}}
		برای استفاده از سرویس 
								{\large \lr{Apple Pay}} 
								از نرم‌افزار
														{\large \lr{Wallet}} 
		که بصورت پیش‌فرض بر روی دستگاه‌های
								{\large \lr{iPhone 6}} 
								و مدل‌های بالاتر نصب شده است و 
																{\large \lr{Apple Watch}} 
							،	 استفاده می‌شود. 
		\begin{figure}[h]
			\centering
			\includegraphics[height=1.5cm]{images/apple-wallet-image}
			\caption{آیکون نرم‌افزار {\footnotesize \lr{Wallet}} }
			\label{wallet-image}
		\end{figure}
	\subsection{روش‌های پرداخت سرویس {\large \lr{Apple Pay}} }
	این سرویس برای پرداخت سه راه را به کاربران ارائه می‌دهد.
	\begin{itemize}
		\item[-] پرداخت درون فروشگاه 
		‌\item[-] پرداخت درون نرم‌افزار
		\item[-]پرداخت درون وب‌سایت
	\end{itemize}

	\subsection{مراحل کار با سرویس {\large \lr{Apple Pay}} }
	برای استفاده از این سرویس در ابتدا باید کارت بانکی که به منظور انجام خرید‌ها استفاده می‌شود را تعریف کنیم. لازم به ذکر است که کارت مورد نظر باید جزء کارت‌های پشتیبانی شده توسط
									{\large \lr{Apple Pay}} 
									 باشد. در شکل \ref{applepayenvironment} محیط نرم‌افزار برای اضافه کردن نشان داده شد است. 
									 
									 
	\begin{figure}[h]
		\centering
		\includegraphics[height=7cm]{images/iphone6-ios9-wallet-applepay-add}
		\caption{محیط نرم‌افزار}
		\label{applepayenvironment}
	\end{figure}
با زدن ضرب‌در بالای صفحه وارد مراحل اضافه کردن کارت بانکی خواهید شد و پس از احراز هویت و تایید آن توسط بانک ، 
								{\large \lr{Apple Pay}}  
								برای انجام خرید‌ها آماده است. 
								
								در ادامه مراحل پرداخت برای سه روش ارائه شده توضیح داده خواهد شد. پرداخت‌های فوق هم از طریق گوشی‌های هوشمند 
																{\large \lr{iPhone }} 
																و هم از طریق 
															{\large \lr{Apple Watch}} 
																								امکان‌پذیر است. 
											
								
	\subsubsection{پرداخت درون فروشگاه} 
	کاربر پس از اتمام خرید در فروشگاه، می‌تواند از این روش پرداخت برای تکمیل خرید خود اقدام کند. پس از  ارائه‌ی صورت حساب توسط صندوق دار، کاربر گوشی و یا 
														{\large \lr{Apple Watch}} 
	 خود را نزدیک به میدان 
									{\large \lr{NFC}}
 ترمینال می‌کند، و سپس اجازه‌ی پرداخت را با استفاده از اثر انگشت خود و یا کلمه‌ی عبور می‌دهد.
 		بر روی  									{\large \lr{Apple Watch}}									 									 									کاربر لبه‌ی انتهایی آن را دو بار لمس می‌کند.
 
 ترمینال‌های مجاز برای استفاده از این روش پرداخت دارای نشان شکل     
 \ref{applepaysymbol}
می‌باشند.
 \begin{figure}[h]
 	\centering
 	\includegraphics[height=2cm]{images/applepaysymbol}
 	\caption{نماد 	{\footnotesize \lr{NFC Apple Pay}}}
 	\label{applepaysymbol}
 \end{figure}

 ترمینال‌های مورد پشتیبانی همانند شکل 
 \ref{apple-pay-whole-foods}
 می‌باشند، که دارای نماد 
 									{\large \lr{Apple Pay}}
 									می‌باشند.
	\begin{figure}[h]
		\centering
		\includegraphics[height=5cm]{images/apple-pay-whole-foods-100526329-large}
		\caption{ترمینال پشتیبانی کننده از {\footnotesize \lr{Apple Pay}}}
		\label{apple-pay-whole-foods}
	\end{figure}
	
	\subsubsection{ پرداخت درون نرم‌افزار   }
	کاربر با استفاده از 
		 									{\large \lr{iPhone}}،
		 									 									{\large \lr{Watch}}،
		 	 									{\large \lr{iPad}} ،
		 	 									می‌تواند از 
		 	 									 									{\large \lr{Apple Pay}}
		 	 									 									به عنوان روش پرداخت برای پرداخت‌های درون نرم‌افزار استفاده کند. برای انجام پرداخت از این راه :
	\begin{enumerate}
		\item لمس دکمه
			 									{\large \lr{Apple Pay}}
			 									  و یا لمس
			 									{\large \lr{Buy with Apple Pay}}.
			 									یا انتخاب
			 									 									{\large \lr{Apple Pay}}
به عنوان روش پرداخت در زمان بازبینی نهایی خرید.	(شکل \ref{buttons})
\begin{figure}[h]
	\centering
	\includegraphics[height=2cm]{images/buttons}
	\caption{آیکون‌های {\footnotesize \lr{Apple Pay}} ‌}
	\label{buttons}
\end{figure}


	\item بازبینی اطلاعات صورت حساب ، اطلاعات تماس و نحوه‌ی ارسال خرید، برای حصول اطمینان از صحت اطلاعات.
	اگر کاربر مایل به پرداخت با کارت متفاوتی غیر از کارت پیش فرض می‌باشد، علامت فلش باید لمس شود.
	\item اگر نیاز به وارد کردن اطلاعات تماس ، صورت حساب و یا نحوه‌ی ارسال محموله می‌باشد، کاربر اطلاعات را وارد می‌کند.
	 									{\large \lr{Apple Pay}} 
	 									این اطلاعات را ذخیره می‌کند تا نیازی به وارد کردن دوباره‌ی آن ها نباشد.
	\item بر روی 
	 									{\large \lr{iPhone}}
	 									و 
	 									 {\large \lr{iPad}}
	 								کاربر با استفاده از اثر انگشت اجازه‌ی پرداخت را می‌دهد.
	 								بر روی  									{\large \lr{Apple Watch}}									 									 									کاربر لبه‌ی انتهایی آن را دو بار لمس می‌کند.
	\end{enumerate}
		
		
	\subsubsection{ پرداخت درون  وب سایت}
	
		
		%%%%%%%%%%%%%%%%%%%%%%%%%%%%%%%%%%%%%%%%%%%%%%%%%
		
		\chapter{تکنولوژی {\Large\lr{Tokenisation}} و زیرساخت‌ها }\label{tokenisation}
					\section{تاریخچه تکنولوژی{\large  \lr {Tokenisation}}}
					در ماه مارس سال ۲۰۱۴ کنسرسیوم 
					{\large \lr{EMV}} 
					\LTRfootnote{Europay MasterCard Visa consortium (EMVco)}
					مشخصات فنی تکنولوژی 
					{\large \lr{Tokenisation}} 
					\LTRfootnote{Technical Framework version 1.0}
					را معرفی کرد، که به دنبال آن شرکت‌های فعال در این زمینه با بهره‌وری از این تکنولوژی سرویس‌های کیف‌پول الکترونیکی خود را معرفی کردند.
					
					
					
					
		%%%%%%%%%%%%%%%%%%%%%%%%%%%%%%%%%%%%%%%%%%%%%%%%%
		\chapter{بررسی نحوه‌ی کار و مقایسه‌ی  سرویس‌های   {\Large\lr{Apple Pay} }  و   {\Large\lr{Android Pay} }}\label{comparison}
		
		\chapter{طراحی و پیاده‌سازی نرم‌افزار پرداخت سمت سرور}\label{serverimplementation}
		
		\chapter{طراحی و پیاده‌سازی نرم‌افزار پرداخت سمت کلاینت}\label{clientimplementation}
		
	
	
	\newpage
	\def\bibname{مراجع}
	\bibliographystyle{ieeetr}
	\bibliography{references}
\end{document}