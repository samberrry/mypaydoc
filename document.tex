%Documentation for Final project
%Language:‌ Persian
\documentclass[oneside]{report}
   \usepackage[
   left = 2.5cm ,
   right = 3.5cm,
   bottom = 2cm,
   ]{geometry}
\usepackage{titlesec}

\usepackage[extrafootnotefeatures]{xepersian}

\settextfont{B Nazanin}
\linespread{1.3}
\setlength{\skip\footins}{2cm}
\setlength{\footnotesep}{0.3cm}

\titleformat{\chapter}[display]
	{\normalfont\Huge\bfseries\centering}
	{\chaptertitlename \thechapter}{20pt}{\Huge\bfseries}
\setlatintextfont{Georgia}

\begin{document}
 
 \section*{چکیده}
\Large \noindent
با توجه به مشکلات امنیتی موجود در زمینه‌ی پرداخت الکترونیکی ، استفاده از راه‌حل‌ کیف پول الکترونیکی به منظور رفع این مشکلات حائز اهمیت است. به سبب اینکه پرداخت‌های امروزی ، اطلاعات کارت بانکی را در معرض خطر قرار می‌دهند، پنهان بودن اطلاعات پرداخت و امن‌تر کردن تراکنش‌ها به علاوه‌ی سهولت پرداخت ،از اهدف اصلی کیف پول‌الکترونیکی می‌باشد.

\noindent
در فصل‌های ابتدایی تکنولوژی‌
	{\large \lr	{Tokenisation}}
مورد مطالعه قرار خواهد گرفت، که نقش اصلی در ایمن‌سازی کیف‌پول‌های الکترونیکی دارد، و در باقی فصل‌ها به بررسی بخش‌های مختلف پروژه‌ی پیاده‌سازی شده پرداخته می‌شود.

\noindent
پروژه‌ی انجام شده، پیاده‌سازی مدلی از پرداخت است که با کمک تکنولوژی
{\large \lr {iBeacon}}
اپل به منظور پیدا کردن موقعیت کاربر مورد استفاده قرار گرفته است.


\vspace*{1cm}
\noindent
\textbf{کلید واژگان: } 
{\large \lr	{Tokenisation}} ، {\large \lr {iBeacon}} 

\pagenumbering{gobble}
{\let\cleardoublepage\clearpage 
	\tableofcontents
}
	
	\chapter{مقدمه}\label{introduction}
		\pagenumbering{arabic}
		امروزه اکثر خرید‌ها و به موجب آن پرداخت‌ها از طریق کارت‌های بانکی صورت می‌گیرد، بهمین خاطر توجه سارقان و کلاهبرداران این حوزه را به خود جلب کرده. سرویس‌های کیف‌پول الکترونیکی این اجازه را به کاربران می‌دهند که کارت‌های بانکی خود را بصورت دیجیتالی در دستگاه‌های دیجیتالی مانند تلفن‌های هوشمند ذخیره کنند. بدین طریق کاربران در زمان پرداخت به جای استفاده از کارت‌های بانکی حقیقی از کارت دیجیتالی ذخیره شده بر روی دستگاه دیجیتالیشان استفاده می‌کنند. تدابیر امنیتی مناسبی برای ایمن کردن اینگونه پرداخت‌ها دیده شده که  مهم‌ترین آن‌ها تکنولوژی  {\large \lr	{Tokenisation}}  می‌باشد.
		
		برای خرید یک کالا و یا یک سرویس مدل‌های مختلفی وجود دارد که کاربران می‌توانند با استفاده از آن‌ها پرداخت را انجام دهند. برای مثال کاربران می‌توانند خریدشان را از طریق ترمینال‌های مجهز به تکنولوژی 
			{\large \lr	{NFC}}
		\LTRfootnote{Near Field Communication}
			  ، و یا از طریق نرم‌افزار‌های پرداخت به طرق مختلف انجام دهند. در ادامه ‌ی این فصل به چند نمونه از مدل‌های پرداخت که در حال حاضر موجود هستند خواهیم پرداخت.		
			\section{تاریخچه تکنولوژی{\large  \lr {Tokenisation}}}
		 در ماه مارس سال ۲۰۱۴ کنسرسیوم 
		{\large \lr{EMV}} 
		\LTRfootnote{Europay MasterCard Visa consortium (EMVco)}
		مشخصات فنی تکنولوژی 
		{\large \lr{Tokenisation}} 
		\LTRfootnote{Technical Framework version 1.0}
		را معرفی کرد، که به دنبال آن شرکت‌های فعال در این زمینه با بهره‌وری از این تکنولوژی سرویس‌های کیف‌پول الکترونیکی خود را معرفی کردند.
		
		از جمله‌ی سرویس‌های کیف‌پول ،می‌توان 
				{\large \lr{Apple Pay}} 
		را نام برد که اپل در ماه سپتامبر ۲۰۱۴ با معرفی 
				{\large \lr{iPhone 6}} 
		آن را عرضه کرد.	شرکت گوگل نیز سرویس 
						{\large \lr{Android Pay}} 
		را در ماه فوریه‌ی ۲۰۱۵ معرفی کرد. سرویس‌های کیف‌پولی که این شرکت‌ها ارائه کرده اند، بر پایه‌ی تکنولوژی 
			{\large \lr	{Tokenisation}}
			می‌باشند، و از این تکنولوژی برای مدیریت کارت‌های بانکی استفاده می‌کنند. لازم به ذکر است که سرویس 
						{\large \lr{Apple Pay}} 
						به عنوان امن‌ترین روش پرداخت موجود در جهان شناخته شده است. 
						\cite{mostsecureBellID}
			در فصل \ref{comparison} به مقایسه‌ی دو سرویس ارائه شده توسط این دو شرکت  و علت برتری سرویس
										{\large \lr{Apple Pay}} 
			پرداخته خواهد شد.
		
		
		
		
		
		
		
		\chapter{تکنولوژی {\Large\lr{Tokenisation}}}\label{tokenisation}

		\chapter{مقایسه ی فنی {\Large\lr{Apple Pay} }  و   {\Large\lr{Android Pay} }}\label{comparison}
	
	
	
	\newpage
	\def\bibname{مراجع}
	\bibliographystyle{ieeetr}
	\bibliography{references}
\end{document}