%Documentation for Final project
%Language:‌ Persian
\documentclass[oneside]{report}
   \usepackage[
   left = 2.5cm ,
   right = 3.5cm,
   bottom = 2.5cm,
   ]{geometry}
\usepackage{titlesec}
\usepackage[perpage]{footmisc}

\usepackage[extrafootnotefeatures]{xepersian}

\settextfont{B Nazanin}
\linespread{1.3}
\setlength{\skip\footins}{2cm}
\setlength{\footnotesep}{0.3cm}

\setcounter{secnumdepth}{4}%bound for subsections
\setcounter{tocdepth}{4}

\titleformat{\chapter}[display]
	{\normalfont\Huge\bfseries\centering}
	{\chaptertitlename \thechapter}{20pt}{\Huge\bfseries}
\setlatintextfont{Georgia}

\begin{document}
 \newcounter{MyC}
\newcommand\addHarfi{\harfi{MyC}\addtocounter{MyC}{1}\ }
\setcounter{MyC}{1}
\pagenumbering{harfi}
 \section*{چکیده}
\Large \noindent
با توجه به مشکلات امنیتی موجود در زمینه‌ی پرداخت الکترونیکی ، استفاده از راه‌حل‌ کیف پول الکترونیکی به منظور رفع این مشکلات حائز اهمیت است. به سبب اینکه پرداخت‌های امروزی ، اطلاعات کارت بانکی را در معرض خطر قرار می‌دهند، پنهان بودن اطلاعات پرداخت و امن‌تر کردن تراکنش‌ها به علاوه‌ی سهولت پرداخت ،از اهدف اصلی کیف پول ‌الکترونیکی می‌باشد.

\noindent
در فصل‌های ابتدایی تکنولوژی‌
	{\large \lr	{Tokenisation}}
مورد مطالعه قرار خواهد گرفت که نقش اصلی در ایمن‌سازی کیف‌پول‌های الکترونیکی را دارد و در باقی فصل‌ها به بررسی بخش‌های مختلف پروژه‌ی پیاده‌سازی شده پرداخته می‌شود.

\noindent
پروژه‌ی انجام شده پیاده‌سازی مدلی از پرداخت است که با کمک تکنولوژی
{\large \lr {iBeacon}}
اپل به منظور پیدا کردن موقعیت کاربر مورد استفاده قرار گرفته است.


\vspace*{1cm}
\noindent
\textbf{کلید واژگان: } 
{\large \lr	{Tokenisation}} ، {\large \lr {iBeacon}} 

%\pagenumbering{gobble}
{\let\cleardoublepage\clearpage 
	\tableofcontents
}
	%%%%%%%%%%%%%%%%%%%%%%%%%%%%%%%%%%%%%%%%%%%%%%%%%%%%%%
	\chapter{مقدمه}\label{introduction}
		\pagenumbering{arabic}
		امروزه اکثر خرید‌ها و به موجب آن پرداخت‌ها از طریق کارت‌های بانکی صورت می‌گیرد، به همین خاطر توجه سارقان و کلاهبرداران این حوزه را به خود جلب کرده است. سرویس‌های کیف‌پول الکترونیکی این اجازه را به کاربران می‌دهند که کارت‌های بانکی خود را بصورت دیجیتالی در دستگاه‌های دیجیتالی مانند تلفن‌های هوشمند ذخیره کنند. بدین طریق کاربران در زمان پرداخت به جای استفاده از کارت‌های بانکی حقیقی از کارت دیجیتالی ذخیره شده بر روی دستگاه دیجیتالیشان استفاده می‌کنند.تدابیر امنیتی مناسبی برای ایمن کردن اینگونه پرداخت‌ها دیده شده که  مهم‌ترین آن‌ها تکنولوژی  {\small \lr	{Tokenisation}}  می‌باشد. در فصل \ref{tokenisation} به توضیح این تکنولوژی پرداخته خواهد شد.
		
		برای خرید یک کالا و یا یک سرویس مدل‌های مختلفی وجود دارد که کاربران می‌توانند با استفاده از آن‌ها پرداخت را انجام دهند. برای مثال کاربران می‌توانند خریدهایشان را از راه ترمینال‌های مجهز به تکنولوژی 
			{\small \lr	{NFC}}
		\LTRfootnote{Near Field Communication}
		به صورت پرداخت غیرتماسی
					  \LTRfootnote{Contacltess payment}
			  ، و یا از راه نرم‌افزار‌های پرداخت به راه‌های مختلف انجام دهند. در ادامه‌ی این فصل به چند نمونه از مدل‌های پرداخت که در حال حاضر موجود هستند خواهیم پرداخت.		
			  
			  لازم به ذکر است ، کیف پول الکترونیکی که به منظور نگهداری بیت کوین‌ها
			  \LTRfootnote{Bitcoin}
			  و یا پول الکترونیکی می‌باشد مورد بحث نخواهد بود.
			  
				\section{معرفی سرویس‌های برجسته‌ی کیف پول الکترونیکی }
		
		از جمله‌ی سرویس‌های کیف‌پول ،می‌توان 
				{\small \lr{Apple Pay}} 
		را نام برد که اپل در ماه سپتامبر ۲۰۱۴ با معرفی 
				{\small \lr{iPhone 6}} 
		آن را عرضه کرد.	شرکت گوگل نیز سرویس 
						{\small \lr{Android Pay}} 
		را در ماه فوریه‌ی ۲۰۱۵ معرفی کرد. سرویس‌های کیف‌پولی که این شرکت‌ها ارائه کرده اند، بر پایه‌ی تکنولوژی 
			{\small \lr	{Tokenisation}}
			می‌باشند، و از این تکنولوژی برای مدیریت کارت‌های بانکی استفاده می‌کنند. لازم به ذکر است که سرویس 
						{\small \lr{Apple Pay}} 
						به عنوان امن‌ترین روش پرداخت موجود در جهان شناخته شده است. 
						\cite{mostsecureBellID}
						بنابراین در ادامه تمرکز بیشتر بر روی توضیح سرویس
							 									 {\small \lr{Apple Pay}}
							 									 خواهد بود، که البته از لحاظ رابط کاربری تفاوت چشم‌گیری با یکدیگر ندارند ، اما از لحاظ فنی جای بحث خواهد داشت.
			در فصل \ref{comparison} به مقایسه‌ی دو سرویس ارائه شده توسط این دو شرکت  و علت برتری سرویس
										{\small \lr{Apple Pay}} 
			پرداخته خواهد شد.
		
		\subsection{ سرویس {\normalsize\lr{Apple Pay}}}
		برای استفاده از سرویس 
								{\small \lr{Apple Pay}} 
								از نرم‌افزار
														{\small \lr{Wallet}} 
		که بصورت پیش‌فرض بر روی دستگاه‌های
								{\small \lr{iPhone 6}} 
								و مدل‌های بالاتر نصب شده است و 
																{\small \lr{Apple Watch}} 
							،	 استفاده می‌شود. 
		\begin{figure}[h]
			\centering
			\includegraphics[height=1.5cm]{images/apple-wallet-image}
			\caption{آیکون نرم‌افزار {\footnotesize \lr{Wallet}} }
			\label{wallet-image}
		\end{figure}
	\subsection{روش‌های پرداخت سرویس {\normalsize \lr{Apple Pay}} }
	این سرویس برای پرداخت سه راه را به کاربران ارائه می‌دهد.
	\begin{itemize}
		\item[-] پرداخت درون فروشگاه 
		‌\item[-] پرداخت درون نرم‌افزار
		\item[-]پرداخت درون وب‌سایت
	\end{itemize}

	\subsection{مراحل کار با سرویس {\normalsize \lr{Apple Pay}} }
	برای استفاده از این سرویس در ابتدا باید کارت بانکی که به منظور انجام خرید‌ها استفاده می‌شود را تعریف کنیم. لازم به ذکر است که کارت مورد نظر باید جزء کارت‌های پشتیبانی شده توسط
									{\small \lr{Apple Pay}} 
									 باشد. در شکل \ref{applepayenvironment} محیط نرم‌افزار برای اضافه کردن نشان داده شد است. 
									 
									 
	\begin{figure}[h]
		\centering
		\includegraphics[height=7cm]{images/iphone6-ios9-wallet-applepay-add}
		\caption{محیط نرم‌افزار{\footnotesize \lr{Apple Pay}}  }
		\label{applepayenvironment}
	\end{figure}
با زدن ضرب‌در بالای صفحه وارد مراحل اضافه کردن کارت بانکی خواهید شد و پس از احراز هویت و تایید آن توسط بانک ، 
								{\small \lr{Apple Pay}}  
								برای انجام خرید‌ها آماده است. 
								
								در ادامه مراحل پرداخت برای سه روش ارائه شده توضیح داده خواهد شد. پرداخت‌های فوق هم از طریق گوشی‌های هوشمند 
																{\small \lr{iPhone }} 
																و هم از طریق 
															{\small \lr{Apple Watch}} 
																								امکان‌پذیر است. 
											
								
	\subsubsection{پرداخت درون فروشگاه} 
	کاربر پس از اتمام خرید در فروشگاه، می‌تواند از این روش پرداخت برای تکمیل خرید خود اقدام کند. پس از  ارائه‌ی صورت حساب توسط صندوق دار، کاربر گوشی و یا 
														{\small \lr{Apple Watch}} 
	 خود را نزدیک به میدان 
									{\small \lr{NFC}}
 ترمینال می‌کند، و سپس اجازه‌ی پرداخت را با استفاده از اثر انگشت خود و یا کلمه‌ی عبور می‌دهد.
 		بر روی  									{\small \lr{Apple Watch}}									 									 									کاربر لبه‌ی انتهایی آن را دو بار لمس می‌کند.
 ترمینال‌های مجاز برای استفاده از این روش پرداخت دارای نشان شکل     
 \ref{applepaysymbol}
می‌باشند.
 \begin{figure}[h]
 	\centering
 	\includegraphics[height=2cm]{images/applepaysymbol}
 	\caption{نماد 	{\footnotesize \lr{NFC Apple Pay}}}
 	\label{applepaysymbol}
 \end{figure}

 ترمینال‌های مورد پشتیبانی همانند شکل 
 \ref{apple-pay-whole-foods}
 می‌باشند، که دارای نماد 
 									{\small \lr{Apple Pay}}
 									هستند.
	\begin{figure}[h]
		\centering
		\includegraphics[height=5cm]{images/apple-pay-whole-foods-100526329-large}
		\caption{ترمینال پشتیبانی کننده از {\footnotesize \lr{Apple Pay}}}
		\label{apple-pay-whole-foods}
	\end{figure}
	
	\subsubsection{ پرداخت درون نرم‌افزار   }
	کاربر با استفاده از 
		 									{\small \lr{iPhone}}،
		 									 									{\small \lr{Watch}}،
		 	 									{\small \lr{iPad}} ،
		 	 									می‌تواند از 
		 	 									 									{\small \lr{Apple Pay}}
		 	 									 									به عنوان روش پرداخت برای پرداخت‌های درون نرم‌افزار استفاده کند. برای انجام پرداخت از این راه :
	\begin{enumerate}
		\item لمس دکمه
			 									{\small \lr{Apple Pay}}
			 									  و یا لمس
			 									{\small \lr{Buy with Apple Pay}}.
			 									یا انتخاب
			 									 									{\small \lr{Apple Pay}}
به عنوان روش پرداخت در زمان بازبینی نهایی خرید.	(شکل \ref{buttons})
\begin{figure}[h]
	\centering
	\includegraphics[height=2cm]{images/buttons}
	\caption{آیکون‌های {\footnotesize \lr{Apple Pay}} ‌}
	\label{buttons}
\end{figure}


	\item بازبینی اطلاعات صورت حساب ، اطلاعات تماس و نحوه‌ی ارسال خرید، برای حصول اطمینان از صحت اطلاعات.
	اگر کاربر مایل به پرداخت با کارت متفاوتی غیر از کارت پیش فرض می‌باشد، علامت فلش باید لمس شود.
	\item اگر نیاز به وارد کردن اطلاعات تماس ، صورت حساب و یا نحوه‌ی ارسال محموله می‌باشد، کاربر اطلاعات را وارد می‌کند.
	 									{\small \lr{Apple Pay}} 
	 									این اطلاعات را ذخیره می‌کند تا نیازی به وارد کردن دوباره‌ی آن ها نباشد.
	\item بر روی 
	 									{\small \lr{iPhone}}
	 									و 
	 									 {\small \lr{iPad}}
	 								کاربر با استفاده از اثر انگشت اجازه‌ی پرداخت را می‌دهد.
	 								بر روی  									{\small \lr{Apple Watch}}									 									 									کاربر لبه‌ی انتهایی آن را دو بار لمس می‌کند.
	\end{enumerate}
		
		
	\subsubsection{ پرداخت درون  وب سایت}
	اپل پشتیبانی از پرداخت درون وب‌سایت خود را به تازگی در پاییز ۲۰۱۵ به همراه مک ‌بوک جدیدش معرفی کرد. پرداخت از این راه فقط از طریق مرورگر 
		 									 {\small \lr{Safari}}
		 									 قابل انجام است. مراحل انجام پرداخت همانند پرداخت درون نرم‌افزار می‌باشد، به جز اینکه در مک‌بوک جدید اپل قابلیت خواندن اثر انگشت اضافه شده، در مک‌بوک‌های سری قبل کاربر مجبور به اتصال گوشی و مک از راه بلوتوث ، و انجام پرداخت از راه گوشی می‌باشد.
		 									 
	\section{‌‌نرم‌افزار‌های برجسته‌ی پرداخت موبایل }		
		بانک‌ها در سراسر جهان در حال عرضه‌ی پلتفورم‌های جدید هستند و شرکت‌های فعال دراین حوزه در حال انتشار در بازار‌های کشورها هستند، مصرف کنندگان نیز اشتیاق استفاده از این محصولات را نشان داده اند. 
	به نقل از 
			 									 {\small \lr{Juniper Research}}
			 									 ، پیش بینی‌شده است که پرداخت‌های صورت گرفته از راه موبایل و پرداخت بدون‌تماس در طی سال ۲۰۱۶  بالغ بر ۳.۶ تریلیون دلار خواهد بود، بنابراین آینده‌ی پرداخت موبایل بسیار امید‌بخش و خوب بنظر می‌آید. 
	\cite{juniperresearch}
			 									 
			 									 با چنین انتظار بالایی ، لیستی از نرم‌افزار‌های ابداعی پرداخت موبایل در بازار کنونی در اینجا آورده شده است. ویژگی برجسته‌ی هر نرم‌افزار که آن‌را از بقیه‌ی نرم‌افزار‌ها متمایز کرده است نیز بیان شده است.
	\subsection{ {\small \lr{Starbucks}}}
		 			 									 ویژگی : طرح‌های وفاداری ، سرویس پرداخت از پیش
	
	 			 				 {\small \lr{Starbucks}}	 			 									 
	 	 همیشه نسبت به رقبا در این عرصه برتری داشته است. از سال ۲۰۰۱ جزء اولین‌ها در ارائه‌ی 
	 	 			 									 {\small \lr{WiFi}} 
	 	 			 									 درون فروشگاه بود و از سال ۲۰۰۹ جزء اولین پذیرندگان پرداخت موبایل بوده است. شمار تراکنش‌های موبایلی در فروش فروشگاه‌ها ۲۵ درصد کل ترانکش‌ها می‌باشد که با استفاده از آسان تر کردن راه‌های پرداخت برای مصرف کنندگان و ارائه‌ی سرویس سفارش از قبل، پرداخت‌ها و دریافت طرح‌های وفاداری و جایزه را آسان‌تر کرده است. 
	 	 			 									 
		\subsection{ {\small \lr{Royal Bank of Canada (RBC)}}}
		ویژگی : کارت‌های هدیه ، رسید‌های دیجیتالی
		
		بزرگ‌ترین  صادرکننده‌ی کارت
				\LTRfootnote{card issuer}
		 کانادا با ۶.۵ میلیون کارت بانکی، بانک شناخته شده‌ی پیشتاز در عرصه‌ی پرداخت می‌باشد. نرم‌افزار این بانک بر پایه‌ی تکنولوژی 			
	 {\small \lr{HCE}} 	\LTRfootnote{Host Card Emulation} 
	 می‌باشد، بدین معنی که کاربر نیازی ندارد حتما سیم‌کارت خاصی را داشته باشد. (در فصل 
	 \ref{tokenisation}
	 در مورد تکنولوژی 
	 {\small \lr{HCE}} 
	 صحبت خواهد شد و ضعف و قوت‌های این تکنولوژی بیان خواهد شد
	 )
	
	\begin{figure}[h]
		\centering
		\includegraphics[height=5cm]{images/RBC}
		\caption{نرم‌افزار موبایل 	{\small \lr{RBC}}}
		\label{fig:rbc}
	\end{figure}
در این نرم‌افزار علاوه بر اینکه کاربر می‌تواند کارت‌‌های اعتباری 
				\LTRfootnote{credit card}
و کارت بدهی 
						\LTRfootnote{debit card}
		را در کیف‌پول ذخیره کند و با استفاده از آن پرداخت درون فروشگاه انجام دهد، می‌تواند کارت‌های جایزه را نیز به کیف‌پول خود اضافه کند و یا کارت‌های جایزه‌ی جدید را از طرح‌های وفاداری مختلف بخرد و اضافه کند، و آن‌ها را به دوستان خود ارسال کند.
		
		\subsection{{\small \lr{ANZ‌ Bank}}}
		\normalsize{
		ویژگی : انتخاب روش پرداخت
		
		بانک 
			 {\small \lr{ANZ} }
			 	استرالیا اخیرا در ضمینه‌ی پرداخت بدون تماس خبرساز بود. نرم‌افزار این بانک روش‌های مختلفی را برای اینکه کاربر چگونه فرآیند پرداخت خود را بر روی دستگاه خود آغاز کند، ارائه می‌دهد. کاربر می‌تواند انتخاب کند که یا دستگاه قبل از اینکه بخواهد پرداخت انجام دهد کاربر آن‌را فقط فعال کند و یا کاربر برنامه را اجرا کند قبل از پرداخت و یا کاربر با وارد کردن یک کد عبور آماده‌ی پرداخت شود. بدین ترتیب نرم‌افزار به کاربر اجازه می‌دهد
			 	که حالتی که برای آن راحت‌تر است را انتخاب کند.
			 	\begin{figure}[h]
			 		\centering
			 		\includegraphics[width=4cm]{images/ANZ-Bank-Select-how-to-Pay}
			 		\caption{نرم‌افزار بانک 	{\footnotesize \lr{ANZ}} }
			 		\label{fig:anz-bank-select-how-to-pay}
			 	\end{figure}
			 			
		\subsection{{\small \lr{CaixaBank}}}
		ویژگی : پرداخت با ابزار‌های پوشیدنی
		
			 {\small \lr{CaixaBank}} 
			 صاحب شهرت در عرصه‌ی پرداخت موبایل ، و بازار بزرگی از پرداخت‌های بدون تماس در کشور اسپانیا می‌باشد. در سال ۲۰۱۵ پرداخت موبایل بر پایه‌ی تکنولوژی 
			  	 			 				 {\small \lr{HCE}}
خود را ارائه کرد، و در سال ۲۰۱۶ 
	 			 {\small\lr{ImaginBank}}	 			 										 			 									 
		را معرفی کرد ، که با استفاده از نرم‌افزار 
			 			 				 {\small \lr{ImaginPay}}	
			 			 				 با بهره‌گیری از تکنولوژی‌
			 			 			 {\small\lr{NFC}}	 	
			 			 			 و گوشی‌های هوشمند و ابزار‌های پوشیدنی ، کاربر اقدام به پرداخت می‌کند.
			 			 			 
			 			 			 \begin{figure}[h]
			 			 			 	\centering
			 			 			 	\includegraphics[height=4cm]{images/CaixaBank-pay-with-wearables-768x432}
			 			 			 	\caption{نرم‌افزار بانک {\footnotesize \lr{CaixaBank}}}
			 			 			 	\label{fig:caixabank-pay-with-wearables-768x432}
			 			 			 \end{figure}
			 			 			 		
		\subsection{{\small \lr{Capital One}}} 			 									 
		ویژگی : امنیت با استفاده از تکنولوژی 
					 			 				 {\small \lr{Tokenisation}}	
 
 
 اواخر سال میلادی ۲۰۱۵ اولین بانک آمریکایی بود که پرداخت از طریق 
 			 			 				 {\small \lr{NFC}}	
 			 			 				 درون نرم‌افزار‌های موبایلش را عرضه کرد. مورد قابل توجه تر این است که این نرم‌افزار از تکنولوژی 
 			 			 				 			 			 				 {\small \lr{Tokenisation}}
 			 			 				 			 			 				 به عنوان مکانیزم امنیتی خود استفاده کرده است تا از اطلاعات کارتی و حساس کاربران استفاده‌ای نشوند و امن بمانند.
 			 			 				 			 			 				 
 \begin{figure}[h]
 	\centering
 	\includegraphics[height=5cm]{images/Capital-One-Tokenization-768x384}
 	\caption{نرم‌افزار بانک {\footnotesize \lr{Capital One}}}
 	\label{fig:capital-one-tokenization-768x384}
 \end{figure}

 			\subsection{{\small \lr{SnapScan}}}
 			ویژگی : استفاده از تکنولوژی 
 			 			 			 				 {\small \lr{Beacon}}	
 			 			 			 			\newline
  			 			 			 				 {\small \lr{SnapScan}}
  			 			 			 				 بیش از ۳۰۰۰۰ پذیرنده \LTRfootnote{Merchant}
  			 			 			 				  را تحت پشتیبانی خود دارد و با تمامی بانک‌های آفریقای جنوبی همکاری می‌کند. این نرم‌افزار از کد‌های دو بعدی 
  			 				\LTRfootnote{QR-code}			 			 				 
برای پیدا کردن فروشگاهی که کاربر در آن حاضر است  در زمان انجام پرداخت استفاده می‌کند. اما به تازگی قابلیتی اضافه کرده است به نام 
 			 			 			 				 {\small\lr{SnpanBeacons}} . 
 			 			 			 				 با استفاده از این قابلیت مکان کاربر بدون استفاده از هیچ‌گونه اسکن و فقط با حضور کاربر در فروشگاه مشخص می‌شود. در فصل 
 			 			 			 				 \ref{clientimplementation}
 			 			 			 				 به معرفی این تکنولوژی و طرز کار آن به طور کامل پرداخته خواهد شد.
 	
	  \section{سرویس{\normalsize \lr{Amazon Go}} }
 		{\small \lr{Amazon Go}} 		
 	نوع جدیدی از فروشگاه  بدون نیاز به چک‌نهایی
 	\LTRfootnote{checkout}می‌باشد.
شرکت 
 		{\small \lr{Amazon}} پیشرفته‌ترین نوع پرداخت در جهان را بوجود آورده به این شکل که نیازی به ماندن در صف برای انجام پرداخت وجود ندارد. 
 	
 	به گفته‌ی آمازون	عرضه‌ی کامل این سرویس اوایل سال ۲۰۱۷ خواهد بود. 
 		
 		طرز استفاده ازین سرویس به این شکل خواهد بود که کاربر از نرم‌افزار 
 		 		{\small \lr{Amazon Go}} 		
 برای ورود به فروشگاه استفاده می‌کند ، سپس خرید خود را در فروشگاه انجام می‌دهد و با خروج از فروشگاه چندی بعد آمازون رسید پرداخت را به کاربر می‌دهد.

آمازون	از تکنولوژی‌هایی که در ماشین‌های خودران استفاده می‌شود بهره می‌برد : 
 					 			{\small \lr{computer vision}} ،
 					 				{\small \lr{deep learning}} ،
 					 					{\small \lr{sensor fusion}}
 					 					 
 \noindent
 آمازون اسم این تکنولوژی را 
 	{\small \lr{Just Walk Out}}
 	گذاشته است.
 			 			 		
 			 			 			 				 			 			 	 
  	\section{معرفی نرم‌‌افزار پیاده‌سازی شده}
   در این بخش به معرفی نحوه‌ی کار و قسمت‌های مختلف نرم‌افزار پیاده‌سازی شده  پرداخته می‌شود. 
   
   		   \subsection{تکنولوژی‌های استفاده شده}
   		 این نرم‌افزار  دارای دو بخش، سمت کلاینت و سمت سرور می‌باشد. نرم‌افزار سمت کلاینت بر روی پلتفورم 
   		 {\small\lr{ios}}
   		 پیاده‌سازی شده است  و قابل استفاده و بر روی تلفن‌های همراه 
   		 {\small\lr{iPhone}}
   		 می‌باشد. در این نرم‌افزار برای پیدا کردن موقعیت فعلی کاربر از تکنولوژی 
   		 {\small\lr{iBeacon}}
   		و برای اضافه کرد کالا‌ها به لیست خرید از 
   		  {\small\lr{QR}} 
   		   		 کد 
   		 استفاده شده است. 
   		 
   		 
   		  نرم‌افزار سمت سرور متشکل‌ از وب‌سرویس‌ها می‌باشد که از پلتفورم جاوا بهره می‌برند.  
   		  برای پیاده‌سازی وب‌سرویس‌ها ،
   		     		 {\small\lr{Servlet Framework}}
   		     		 و برای ارتباط با پایگاه‌داده  
   		     		    		 {\small\lr{Hibernate Framework}}
   		     		    		 مورد استفاده قرار گرفته است.
   		  همچنین برای تایید شماره‌ی تلفن همراه کاربر ,از  سرویس پیام‌کوتاه استفاده شده است.
   		  
		   \subsection{نحوه‌ی کار نرم‌اافزار}

مراحل کار با نرم‌افزار به شکل زیر می‌باشد
\begin{enumerate}
	\item در صورت اینکه کاربر از قبل دارای حساب کاربری باشد کاربر تنها اقدام به ورود می‌کند،
	  			 				\LTRfootnote{sign-in}			 
	در غیر این صورت اقدام به ثبت نام کرده. بعد از ثبت‌نام کاربر پیامکی حاوی یک کد چهار رقمیمبنی  بر تایید شماره‌ی تلفن دریافت می‌کند ، سپس آن‌را وارد می‌کند.
	\item قبل از انجام خرید کاربر به منظور تکمیل فرآیند خرید نیازمند به فراهم کردن اطلاعات کارت بانکی می‌باشد. با لمس کردن اضافه کردن کارت کاربر وارد صفحات اضافه کردن کارت بانکی می‌شود و این عملیات را به انجام می‌رساند.
	\item با حضور کاربر در فروشگاه‌های تحت پشتیبانی ، نرم‌افزار ،مکان فعلی کاربر را تشخیص داده و اقدام به خرید می‌کند. در صورتی که فروشگاه تحت پشتیبانی در نزدیکی کاربر نباشد امکان انجام خرید نخواهد بود.
	\item با تشخیص مکان کاربر توسط نرم‌افزار ، امکان اسکن کردن کد‌های کالاها فراهم می‌شود، بدین منطور کاربر با استفاده از دوربین تلفن همراه کد
	   		     		 {\small\lr{QR}}
	  			 				 را اسکن می‌کند.
	\item اطلاعات کالا بعلاوه‌ی مبلغ کالا به کاربر نمایش داده می‌شود.
	\item کاربر اضافه به لیست خرید را لمس می‌کند و کالا به لیست خرید اضافه می‌شود.
	\item در صفحه‌ی لیست کالاها ، کاربر پرداخت در اینجا را می‌زند و وارد مراحل پرداخت می‌شود
\end{enumerate}
		   
		%%%%%%%%%%%%%%%%%%%%%%%%%%%%%%%%%%%%%%%%%%%%%%%%%
		
		\chapter{تکنولوژی {\Large\lr{Tokenisation}} و زیرساخت‌ها }\label{tokenisation}
					\section{تاریخچه تکنولوژی{\large  \lr {Tokenisation}}}
					در ماه مارس سال ۲۰۱۴ کنسرسیوم 
					{\small \lr{EMV}} 
					\LTRfootnote{Europay MasterCard Visa consortium (EMVco)}
					مشخصات فنی تکنولوژی 
					{\small \lr{Tokenisation}} 
					\LTRfootnote{Technical Framework version 1.0}
					را معرفی کرد، که به دنبال آن شرکت‌های فعال در این زمینه با بهره‌وری از این تکنولوژی سرویس‌های کیف‌پول الکترونیکی خود را معرفی کردند.
		
					
					
					
					
		%%%%%%%%%%%%%%%%%%%%%%%%%%%%%%%%%%%%%%%%%%%%%%%%%
		\chapter{بررسی نحوه‌ی کار و مقایسه‌ی  سرویس‌های   {\Large\lr{Apple Pay} }  و   {\Large\lr{Android Pay} }}\label{comparison}
		در این فصل به نحوه‌ی کار سرویس 
							{\small \lr{Apple Pay}} 
		و 
							{\small \lr{Android Pay}}
	از دید فنی خواهیم پرداخت ،و چگونگی استفاده‌ی هر دو سرویس از تکنولوژی 
						{\small \lr{Tokenisation}} 
	بررسی خواهد شد. به عنوان مطرح‌ترین کیف پول الکترونیکی ،سرویس 
						{\small \lr{Apple Pay}}
 بطور کامل مورد بررسی خواهد گرفت و در انتها به مقایسه‌ی دو سرویس و مزایا و معایب هر کدام پرداخته می‌شود.
 
 \section{{\large\lr{Apple Pay}} }
	 با استفاده از
	 						{\small \lr{Apple Pay}}
کاربران می‌توانند با  دستگاه‌های 
{\small \lr{iOS}}
	پشتیبانی شده و 
	{\small \lr{Apple Watch}}
	اقدام به پرداخت امن و راحت بکنند. 
انجام خرید توسط این سرویس بسیار ساده است و برای تأمین امنیت آن از قابلیت‌های نرم‌افزاری و سخت‌افزاری استفاده شده است.

{\small \lr{Apple Pay}}
همچنین برای محافظت از اطلاعات شخصی کاربران نیز طراحی شده است. اطلاعات تراکنش‌های انجام شده توسط 
{\small \lr{Apple Pay}}
ذخیره نمی‌شوند و این اطلاعات فقط بین کاربر پذیرنده و صادر کننده کارت رد‌‌ و بدل خواهد شد.

\subsection{اجزای {\normalsize \lr{Apple Pay}} }
 
 \subsubsection{{\small \lr{Secure Element}}}
 {\small \lr{Secure Element}}
 یک استاندارد صنعتی و یک تراشه‌ی گواهی شده
 					\LTRfootnote{certified Chip}
 مي‌باشد که تحت پلتفورم 
 {\small \lr{Java Card}}
 است و منطبق با نیازمندی‌های صنعت مالی برای پرداخت‌های الکترونیکی می‌باشد.
 
  \subsubsection{{\small \lr{NFC Controller}}}
  {\small \lr{NFC Controller}}
  پروتکل‌های مربوط به 
   {\small \lr{NFC}}
   را مدیریت می‌کند و ارتباط دهنده بین پردازنده‌ی نرم‌افزار 
    					\LTRfootnote{Application Processor}
  و 
   {\small \lr{Secure Element}}
   ، و همچنین ارتباط دهنده بین 
    {\small \lr{Secure Element}}
    و ترمینال
     {\small \lr{POS}} 
         					\LTRfootnote{Point-of Sale terminal}
      می‌باشد.
      
        \subsubsection{{\small \lr{Wallet}}}
        از
           {\small \lr{wallet}}
           برای اضافه کردن و مدیریت کارت‌های اعتباری، بدهی و جایزه و همچنین انجام پرداخت‌ها از راه سرویس 
              {\small \lr{Apple Pay}}
              استفاده می‌شود.
              کاربران می‌توانند اطلاعات مربوط به کارت خود ، صادر کننده‌ی کارت ، تراکنش‌های انجام شده‌ی اخیر و سیاست‌های حریم‌شخصی 
         					\LTRfootnote{privacy policy}              
              بانک خود را مشاهده کنند.
 
         \subsubsection{{\small \lr{Secure Enclave}}}
    بر رو
               {\small \lr{iPhone}} و
                          {\small \lr{iPad}}
                          ، 
                                     {\small \lr{Secure Enclave}}
    فرآیند احراز هویت کاربر را برعهده دارد ، و قادر می‌سازد که یک تراکنش صورت پذیرد و داده‌ی اثرانگشت مورد استفاده برای تکنولوژی 
               {\small \lr{TouchID}}
               را فراهم می‌کند. 
               
   بر روی 
              {\small \lr{Apple Watch }}
              ، دستگاه ممکن است نیاز به باز کردن قفل داشته باشد، و کاربر باید کنار انتهای آن‌را دوبار کلیک کند. این عمل دوبار کلیک کردن تشخیص داده می‌شود و به 
                         {\small \lr{Secure Element}}
                      مستقیما   ارسال می‌شود ، بدون اینکه از پردازنده‌ی نرم‌افزار عبور کند.
                      
      \subsubsection{سرورهای{\small \lr{Apple Pay }}} 
      سرورهای 
   {\small \lr{Apple Pay}}
   وضعیت کارت‌های اعتباری و بدهی در 
   {\small \lr{Wallet}}
   و 
   {\small \lr{Device Account Number}}
   در 
   {\small \lr{Secure Elemetn}}
   را مدیریت می‌کنند. این سرورها هم با دستگاه‌‌های کاربران و هم با سرورهای شبکه‌ی پرداخت ارتباط برقرار می‌کنند.
   سرور‌های 
   {\small \lr{Apple Pay}}
   همچنین مسئول رمزنگاری دوباره‌ی گواهی‌های پرداخت 
         					\LTRfootnote{Payment Credentials}      
         					برای پرداخت درون نرم‌افزار           
   هستند. 
               
    \subsection{چگونه {\small \lr{Apple Pay }} از {\small \lr{Secure Element }} استفاده می‌کند }           
          
      
               
               
               
 
		
		%%%%%%%%%%%%%%%%%%%%%%%%%%%%%%%%%%%%%%%%%%%%%%%%
		
		\chapter{طراحی و پیاده‌سازی نرم‌افزار پرداخت سمت سرور}\label{serverimplementation}
		
		\chapter{طراحی و پیاده‌سازی نرم‌افزار پرداخت سمت کلاینت}\label{clientimplementation}
		
	
	
	\newpage
	\def\bibname{مراجع}
	\bibliographystyle{ieeetr}
	\bibliography{references}
\end{document}