\documentclass{book}
   \usepackage[
   left = 2.5cm ,
   right = 3.5cm,
   ]{geometry}
\usepackage{titlesec}
\usepackage{amsthm,amsmath}

\usepackage[extrafootnotefeatures]{xepersian}

\settextfont{B Nazanin}
\linespread{1.3}

\titleformat{\chapter}[display]
	{\normalfont\Huge\bfseries\centering}
	{\chaptertitlename \thechapter}{20pt}{\Huge\bfseries}
	
%%%%%%%%%%%%%%%%%%%%%%%%%%%%%
\makeatletter
\renewcommand  \thechapter{\hspace*{0.0001cm} \@tartibi\c@chapter}
\renewcommand*\l@chapter[2]{%
	\ifnum \c@tocdepth >\m@ne
	\addpenalty{-\@highpenalty}%
	\vskip 1.0em \@plus\p@
	\setlength\@tempdima{1.5em}%
	\begingroup
	\parindent \z@ \rightskip \@pnumwidth
	\parfillskip -\@pnumwidth
	\leavevmode \bfseries
	\advance\leftskip\@tempdima
	\hskip -\leftskip
	فصل 
	#1 \nobreak\hfil \nobreak\hb@xt@\@pnumwidth{\hss #2}\par
	\penalty\@highpenalty
	\endgroup
	\fi
} 
\makeatother


\begin{document}
 
 \section*{چکیده}
\Large \noindent
با توجه به مشکلات امنیتی موجود در زمینه‌ی پرداخت الکترونیکی ، استفاده از راه‌حل‌ کیف پول الکترونیکی به منظور رفع این مشکلات حائز اهمیت است. به سبب اینکه پرداخت‌های امروزی ، اطلاعات کارت بانکی را در معرض خطر قرار می‌دهند، پنهان بودن اطلاعات پرداخت و امن‌تر کردن تراکنش‌ها به علاوه‌ی سهولت پرداخت ،هدف اصلی کیف پول‌الکترونیکی می‌باشد.

\noindent
در ابتدا بطور مفصل تکنولوژی‌
	{\large \lr	{Tokenisation}}
مورد مطالعه قرار خواهد گرفت، که نقش اصلی در ایمن‌سازی کیف‌پول‌های الکترونیکی دارد، و سپس به بررسی بخش‌های مختلف پروژه‌ی پیاده‌سازی شده پرداخته می‌شود.

\noindent
پروژه‌ی انجام شده، پیاده‌سازی روشی از پرداخت است که با کمک تکنولوژی
{\large \lr {iBeacon}}
اپل به منظور پیدا کردن مکان کاربر مورد استفاده قرار گرفته است.


\vspace*{1cm}
\noindent
\textbf{کلید واژگان: } 
{\large \lr	{Tokenisation}} ، {\large \lr {iBeacon}} 

\pagenumbering{gobble}
{\let\cleardoublepage\clearpage 
	\tableofcontents
}
	
	\chapter{مقدمه}\label{chap1}
		\pagenumbering{arabic}
	
\end{document}